\documentclass{article}
\usepackage[utf8]{inputenc}
\usepackage{mathtools}
\usepackage{titlesec}
% \usepackage{amsmath,amssymb}
\usepackage{amsfonts,amsmath,amssymb}
\usepackage{cleveref}
\usepackage{bm}
\usepackage{algorithm}
\usepackage{algpseudocode}
\usepackage{pifont}
\usepackage{color}
\usepackage[a4paper, total={6in, 8in}]{geometry}

% TODO
% check that \*z_k = (z_{k,1} ... z_{k,J}) is introduced correctly
% handle initial states in M-Step section




\crefname{section}{§}{§§}

\DeclareMathOperator{\E}{\mathbb{E}}
\DeclareMathOperator{\Pp}{\mathbb{P}}
\DeclareMathOperator*{\argmax}{arg\,max}


\newcommand{\g}{\mathbf{g}}
\newcommand{\m}{\mathbf{m}}
\newcommand{\vv}{\mathbf{v}}
\newcommand{\n}{\mathbf{n}}
% \newcommand{\*x}{\mathbf{x}}
\newcommand{\z}{\mathbf{z}}
\newcommand{\G}{\mathbf{G}}
\newcommand{\Hh}{\mathbf{H}}
% \newcommand{\*x}{\mathbf{X}}
\newcommand{\W}{\mathbf{W}}
\newcommand{\V}{\mathbf{V}}
\newcommand{\data}{\mathcal{D}}
\newcommand{\tr}{\text{T}}

\def\code#1{\texttt{#1}}
\def\*#1{\mathbf{#1}}



\title{Spectrogram Estimation for Spiking Data}
\author{Andrew Song, John Tauber}
\date{2021}

\begin{document}

\maketitle

\section{Notation}
The goal of this document will be to outline a generative model and inference
framework for estimating a spectrogram from spiking observations. This will
serve as the basis for combining with lfp data. The notation is as follows:

\begin{align*}
   &\text{window } k = 1, \ldots, K \\
   &\text{timebin } j = 1, \ldots, J \text{ time within window K} \\
   &n^c_{k,j} \in \{0,1\} := \text{spike data at time $j$ within window $k$ for neuron $c$} \\
   &\lambda^c_{k,j} := \text{CIF at time $j$ in window $k$ for neuron c} \\
   &\W \in \mathbb{C}^{J \times J} := \text{inverse-DFT matrix} \\
   &\*x^{\text{T}} := \text{transpose of vector } \*x \\
   &\*z_k\in\mathbb{C}^J := (z_{k,1} \ldots, z_{k,J} )^{\text{T}} 
      \text{as in \cite{Kim2018}}\\
   &\mathcal{L}(\cdot \mid \cdot) - \text{emphasizes likelihood function}
\end{align*}

\section{Generative Model}
Assume we have a time series of spiking observations of length $T$ consisting of
data from $C$ neurons. We divide the data into $K$ windows of length $J$, such
that $T = KJ$.  We denote the full set of
spiking observations as 
\begin{equation*}
   \data = \{n^c_{k,j}\}_{k=1, j=1, c=1}^{K,J,C}
\end{equation*}

We assume as our model for $\data$ a discrete-time approximation to a
point process, where the conditional intensity function (CIF) for neuron $c$ at time $t = k+j$
is a logistic function of a latent process $X_t$
\begin{equation}
   \lambda^c_{k,j} = \frac{1}{1 + \exp(-(\mu^c + \beta^c x_{k,j}))}
\end{equation}
such that, for $c = 1, \ldots, C$: 
\begin{equation}
   n^c_{k,j} \sim \text{Bernoulli}(\lambda^c_{k,j})
\end{equation}

We assume that in each window $k$ the latent process $X_k,j$ is second-order
stationary. Letting $\*x_k$ denote $(x_{k,1}, \ldots, x_{k,J})^{\text{T}}$, and
following the approach of \cite{Kim2018}, we let: 
\begin{equation}
   \*x_k = \W \*z_k
\end{equation}
where $\W$ is an inverse spectral transformation matrix (see Appendix B1), and
$\*z_k$ is the random walk process across the windows:
\begin{equation}
   \*z_k = \*z_{k-1} + \nu_k \quad \quad \nu_k \sim \mathcal{N}(0, 
      \operatorname{diag}(\sigma_{\nu,1}^2,\ldots, \sigma_{\nu,J}^2))
\end{equation}

Note that without the prior $\*z_k = \*z_{k-1} + \nu_k$, we
could estimate the $\*z_k$ independently across the windows as in
\cite{Miran2017}. However, we would like to have the stochastic continuity
constraint, as this provides denoising of the spectrogram - when estimating a
spectrogram in a specific window, the algorithm effectively pools together
estimates from neighboring windows~\cite{Kim2018}. We also add there are
different ways to encode stochastic continuity, such as through integrated
Wiener Process (IWP)~\cite{Song18}. This completes the setup of the generative
model. 

\section{Inference}
Let $\theta = \{\sigma^2_{\nu,j}, \mu^c, \beta^c \}_{j=1,c=1}^{J,C}$. For
inference, we are interested in 1) parameter estimation of $\theta$ to obtain
$\widehat{\theta}$ and 2) computation of the posterior distribution $\Pp
(\*z_k\mid \data, \widehat{\theta})$. The complete-data likelihood can be
expressed as:
\begin{align}
   \log \mathcal{L}(\theta) &= \log \Pp(\data, \*z_{1:K} \mid \theta) \nonumber \\
   &= \log \Pp (\data \mid \*z_{1:K}, \theta) + \log \Pp(\*z_{1:K} \mid \theta),
\end{align}
which under our generative model is:
\begin{align}
      \sum_{k=1}^K & \sum_{c=1}^C \sum_{j=1}^J  n_{k,j}^c (\mu^c + \beta^c (\text{W}\*z_k)_j)
      - \log \big(1 + \exp( \mu^c + \beta^c (\text{W} \*z_k)_j) \big) \nonumber \\
   &-(K\sum_{j=1}^J\log(\pi \sigma^2_{\nu,j}) - \sum_{k=1}^K \sum_{j=1}^J \frac{\mid \mid z_{k,j}  - 
      z_{k-1,j} \mid \mid^2}{\sigma_{\epsilon, j}^2} 
\end{align}
Because we do not observe the $\*z_k$'s, however, we will look to
expectation maximization for a solution, which leads us to a state space model
construction. 

However, the point-process likelihood (using either Poisson or binomial forms)
presents a challenge, since the Gaussian prior on $\*z_k$ is not a conjugate
prior to the likelihood. For this reason, we don't have access to the analytical
form for the posterior distribution. A few solutions exist. First, one could use
data-augmentation scheme via Polya-Gamma data augmentation~\cite{Polson13} and
obtain Monte Carlo (MC) samples from the exact posterior distribution. With MC
samples, one can approximate the required expectations in the E-step, leading to
Monte Carlo EM (MCEM)~\cite{Zhang18}.\\\\
The second option is to be make a Laplace approximation to the binomial
likelihood. The naive approach (Laplace approximation to the entire data) would
be computationally expensive, as this results in a covariance matrix of dimension
$KJ\times KJ$. Rather, we make the following observations: 1) if the
observations were instead complex Gaussian instead of Binomial, the problem
would be reduced to a frequency domain Kalman filter (FDKF) as in
\cite{Kim2018}, and 2) for a given window $k$, we can use a Laplace
approximation to the point process spectra as in \cite{Miran2017}. See Appendix
B2 for details. 
\newline

\section{Algorithm}


At the $(s+1)^{\text{th}}$ iteration of the EM algorithm, we want to take the
expectation of the complete data log-likelihood $\log \mathcal{L}(\theta)$ with
respect to the posterior distribution $\Pp(\*z_{1:K}|\data,\theta^{(s)})$. We
thus have
\begin{align}
   \E [\log \mathcal{L}(\theta)\,|\,\data,\theta^{(s)}] &= \E [\log \Pp(\data, \*z_{1:K} \mid \theta)\,|\,\data,\theta^{(s)}] \nonumber \\
   &= \E [\log \Pp (\data \mid \*z_{1:K}, \theta) + \log \Pp(_{1:K} \mid \theta)\,|\,\data,\theta^{(s)} ], \nonumber
\end{align}
which under our model is
\begin{align}\label{eq:qfunc}
   \E [\log \mathcal{L}(\theta)|\data,\theta^{(s)}]=\E & \Bigg[ \sum_{k=1}^K \sum_{j=1}^J \sum_{c=1}^C n_{k,j}^c (\mu^c + \beta^c (\text{W}\*z_k)_j) 
      - \log \big(1 + \exp( \mu^c + \beta^c (\text{W} \*z_k)_j ) \big) \nonumber \\  
   & -(K\sum_{j=1}^J\log(\pi \sigma^2_{\nu,j}) - \sum_{k=1}^K \sum_{j=1}^J \frac{\mid \mid z_{k,j}  - 
      z_{k-1,j}  \mid \mid^2}{\sigma_{\nu, j}^2} \,\Big|\,\data,\theta^{(s)}\Bigg].
\end{align}
The following quantities are needed for computing the expectation in
Eq.~(\ref{eq:qfunc})
\begin{equation}\label{eq:exps}
   \begin{split}
   \*z_{k \mid K}  &= \E [ \*z_k \mid 
      \data, \theta^{(s)}] \\
   \Lambda_{k \mid K} &= \E [ || \*z_k ||^2 
      \mid \data, \theta^{{(s)}}] \\
   \Lambda_{k, k-1 \mid K} &= \E [ \*z_k \*z^*_{k-1}  
      \mid \data, \theta^{{(s)}}]
   \end{split}
\end{equation}
\subsection{Filter}
These quantities can be efficiently computed using a Kalman filtering and
smoothing algorithm. Let us focus primarily on the Kalman filtering, as the
Kalman smoother is straightforward once we compute the prediction/update
quantities in the filter step. First the one-step prediction for filtering is
simply given as
\begin{equation}
    \begin{split}
    \*z_{k \mid k-1} &= \*z_{k-1 \mid k-1}  \\
   \sigma^2_{k \mid k-1,j} &= \sigma^2_{k-1 \mid k-1, j} + \sigma^2_{\nu, j}. \\
   \end{split}
\end{equation}
The one-step update requires more attention, due to the fact that $\Pp
(\data_k|\*z_k)$ is not a Gaussian distribution. To this end, we perform
Gaussian approximation, i.e., Laplace approximation, to the posterior
distribution $\Pp (\*z_k \mid \data_{1:k},\theta)$, such that $\Pp (\*z_k \mid
\data_{1:k},\theta)\sim
\mathcal{N}(\*z_{k|k},\operatorname{diag}(\sigma_{k|k,1}^2,\ldots,\sigma_{k|k,J}^2))$~\cite{Smith03}.\\

Using Bayes's rule on $\Pp (\*z_k \mid \data_{1:k})$ (we drop $\theta$ for notational simplicity), we have
\begin{equation}\label{eq:kf_bayes}
    \begin{split}
   \Pp ( \*z_k \mid \data_{1:k}) & \propto\int\Pp(\data_k,\*z_k, \*z_{k-1}|\data_{1:k-1})\,d(\*z_{k-1})\\
   & =
      \int\Pp (\data_k \mid\*z_k) 
      \Pp (\*z_k \mid\*z_{k-1})\Pp(\*z_{k-1}| \data_{1:k-1}) \,d(\*z_{k-1}) \\
  & =
      \Pp (\data_k \mid\*z_k) \underbrace{\int
      \Pp (\*z_k \mid\*z_{k-1})\overbrace{\Pp(\*z_{k-1}| \data_{1:k-1})}^{\data(\*z_{k-1|k-1},\Sigma_{k-1|k-1})} \,d(\*z_{k-1})}_{\Pp (\*z_k \mid \data_{1:k-1})=\data(\*z_{k|k-1},\Sigma_{k|k-1})} \\
   &= \Pp (\data_k \mid \*z_k) \Pp (\*z_k \mid \data_{1:k-1}).
      \end{split}
\end{equation}
Note that all conditional independence relations follow from our generative
model. With Eq.~(\ref{eq:kf_bayes}), we can proceed with the Laplace
approximation as follows
\begin{equation}
    \begin{split}
      \*z_{k|k}&=\argmax_{\*z_k}\log \Pp(\*z_k | \data_{1:k})\\
      &=\argmax_{\*z_k}\{\log\Pp (\data_k \mid \*z_k) +\log\Pp (\*z_k \mid \data_{1:k-1})\}\\
      &= \argmax_{\*z_k} \Big\{ \sum_{c=1}^C \sum_{j=1}^J n_{k,j}^c ( \mu^c + \beta^c (\W \*z_k)_j) 
         - \log \big(1 + \exp ( \mu^c + \beta^c(\W \*z_k )_j ) \big)\\
      &\quad \quad \quad \quad \quad \quad - \sum_{j=1}^J \frac{|| z_{k,j} - 
         z_{k|k-1,j}||^2} {\sigma^2_{\nu,j}}\Big\}.
    \end{split}
\end{equation}
To compute $\Sigma_{k|k}$, we compute Hessian of both sides. The full details are given in \ref{Newton}

% TODO Update for pp params
\subsection{Newton's Method for State Update}
\begin{algorithm}
\caption{Newton's Method for Window $k$ Gaussian Approximation}
\label{Newton}
\begin{algorithmic}[1]
\State $\z^{(0)} = \mathbf{0}; \quad i=0$
\State $\bar{n} = \sum_{c=1}^C n^c_{k,j}$
\While {$\text{crit} = 0$}
\State $i \leftarrow i + 1$
\State $\*x = \W \z^{(i-1)}$
\State $\lambda = \frac{1}{1 + e^{\*x}}$
\State $\vv = \Big(\frac{z_1^{(i-1)} - \*z_{k-1 \mid k-1}(\omega_1)}{\sigma^{2(s)}_{\nu,1}},
   \cdots, \frac{z_J^{(i-1)} - \*z_{k-1 \mid k-1}(\omega_J)}{\sigma^{2(s)}_{\nu,J}} \Big)^{\tr}$
\State $\g = C\W^{\tr}(\bar{\n} - \lambda) - \vv \quad$ 
\State $\V = \text{diag}\Big( \frac{1}{\sigma^{2(s)}_{\nu,1}}, \cdots, 
   \frac{1}{\sigma^{2(s)}_{\nu,J}}\Big)$
\State $\G = \text{diag}\Big(\frac{e^{x_1}}{(1+e^{x_1})^2}, \cdots, \frac{e^{x_K}}{(1+e^{x_K})^2} \Big) $
\State $\Hh = -C \W^{\tr} \G \W - \V \quad$ 
\State $\z^{(i)} = \z^{(i-1)} - \Hh^{-1}\g$
\EndWhile
\State $\*z^{(s+1)}_{k \mid k} = \z^{(i)}$
\State $\*x = \W \z^{(1)}$
\State $\G = \text{diag}\Big(\frac{e^{x_1}}{(1+e^{x_1})^2}, \cdots, \frac{e^{x_K}}{(1+e^{x_K})^2} \Big) $
\State $\Hh = -C \W^{\tr} \G \W - \V \quad$ 
\State $\sigma^{2(s+1)}_{k \mid k, j} = - \frac{1}{\Hh_{(j,j)}}$
\end{algorithmic}
\end{algorithm}

\subsection{Smoother}
The smoothing algorithm is performed as in \cite{Kim2018}:
\begin{equation}
   \begin{split}
      G_{k,j} &= \sigma^2_{k \mid k, j} (\sigma^2_{k + 1\mid k, j})^{-1} \\
      z_{k \mid K,j} &= z_{k \mid k, j} 
         + G_{k,j} ( z_{k+1 \mid K, j} - z_{k+1 \mid k,j} ) \\
      \sigma^2_{k \mid K,j} &= \sigma^2_{k\mid k,j} + G^2_{k,j}(\sigma^2_{k+1 \mid K} 
         - \sigma^2_{k+1 \mid k, j})
   \end{split}
\end{equation}

% TODO update/understand sigmas (add summation over j) to Q func
% TODO check sigma MLE with new babadi paper
\subsection{M-Step}
In the M-Step, we maximize the expected log-likelihood of Eq.~(\ref{eq:qfunc}),
using the values computed in Eqs.~(\ref{eq:exps}). Omitting the initial state
and variance for now, for a given frequency $j$ we have:
% TODO - need to add initial state 
\begin{equation}\label{eq:Q}
   \begin{split}
      Q(\theta \mid \theta^{(s)}) &= \E [\log \mathcal{L}_j(\theta)|\data,\theta^{(s)}] \\
      &= \sum_{k=1}^K \sum_{j=1}^J \sum_{c=1}^C n_{k,j}^c (\mu^c + \beta^c (\text{W}\*z_{k \mid K})_j) \\
      &- \sum_{k=1}^K \sum_{j=1}^J \sum_{c=1}^C \E \Big[ \log \big(1 + \exp( \mu^c + \beta^c (\text{W} \*z_k)_j ) \big) \Big ]\\
      &- \sum_{j=1}^J (K)\log(\sigma^2_{\nu,j}) - \frac{1}{\sigma^2_{\nu,j}}
      \sum_{k=1}^K \Big(\Lambda_{k \mid K,j} + -2\Lambda_{k,k-1 \mid K, j} + \Lambda_{k-1 \mid K, j}\Big) %\\
      % & + \text{const.}
   \end{split}
\end{equation}
% TODO write out expectations in (8) as is done in Abbas
% ok, approach here: let SSMT code/paper guide this
For the process variances we calculate a closed-form maximum likelihood estimate (MLE):
% TODO fix for initial state
\begin{equation}\label{eq:sigma_mle}
   \hat{\sigma}^2_{\nu,j} = \frac{1}{K} \sum_{k=1}^K \Big(\Lambda_{k \mid K,j} + 
   -2\Lambda_{k,k-1 \mid K, j} + \Lambda_{k-1 \mid K, j}\Big) \\
\end{equation}
where the values in Eqs.~(\ref{eq:exps}) are computed as
\begin{equation}\label{exps_est}
   \begin{split}
      \Lambda_{k\mid K,j} &= z_{k \mid K,j}^2  + \sigma^2_{k \mid K,j} \\
      \Lambda_{k,k-1 \mid K, j} &= z_{k \mid K,j}  z^*_{k-1 \mid K,j} 
         + G_{k-1,j} \sigma^2_{k\mid K,j}
   \end{split}
\end{equation}

The remaining expectation in Eq.~(\ref{eq:Q}) does not have a simple closed form
solution. In order to find estimates for $\mu^c$ and $\beta^c$, we proceed with
a Monte-Carlo EM procedure to approximate the expectation and find the MLE
numerically. We maximize the function $R$:
\begin{equation}
   \begin{split}
      R(\mu^1, \beta^1, \ldots, \mu^C, \beta^C) &= \sum_{k=1}^K \sum_{j=1}^J \sum_{c=1}^C n_{k,j}^c (\mu^c + \beta^c (\text{W}\*z_{k \mid K})_j) \\
      &- \sum_{k=1}^K \sum_{j=1}^J \sum_{c=1}^C \sum_{m=1}^M \log \big(1 + \exp( \mu^c + \beta^c (\text{W} \*z_k^{(m)})_j ) \big) \\
      \simeq \sum_{k=1}^K \sum_{j=1}^J \sum_{c=1}^C n_{k,j}^c &(\mu^c + \beta^c (\text{W}\*z_{k \mid K})_j) 
         - \sum_{k=1}^K \sum_{j=1}^J \sum_{c=1}^C \E \Big[ \log \big(1 + \exp( \mu^c + \beta^c (\text{W} \*z_k)_j ) \big) \Big ]\\
   \end{split}
\end{equation}

\noindent where $\*z_1^{(m)}, \ldots, \*z_K^{(m)}$ are draws from the posterior
distribution over $\*z_1, \ldots, \*z_K$ computed during the E-Step. The gradients are computed using
\begin{equation}
   \begin{split}
      \frac{\partial R}{\partial \mu^c} &= \sum_{k=1}^K \sum_{j=1}^J n_{k,j}^c 
         - \sum_{k=1}^K \sum_{j=1}^J \sum_{m=1}^M \frac{\exp( \mu^c + \beta^c (\text{W} \*z_k^{(m)})_j )}{1 + \exp( \mu^c + \beta^c (\text{W} \*z_k^{(m)})_j ) } \\
      \frac{\partial R}{\partial \beta^c} &= \sum_{k=1}^K \sum_{j=1}^J n_{k,j}^c(\text{W} \*z_{k \mid K})_j 
         - \sum_{k=1}^K \sum_{j=1}^J \sum_{m=1}^M \frac{(\text{W} \*z_k^{(m)})_j \exp( \mu^c + \beta^c (\text{W} \*z_k)_j )}{1 + \exp( \mu^c + \beta^c (\text{W} \*z_k^{(m)})_j ) } \\
   \end{split}
\end{equation}

% TODO 



% I am unsure about the form / derivation of Eqs.~(\ref{exps_est}), although it seems to be what is
% implemented in the code for \cite{Kim2018}. Also, note that in \cite{Kim2018},
% they use Re$(\Lambda_{k,k-1 \mid K, j})$. I have been unable to figure out why
% that is the case, or what ramifications that may have on our implementation.  

\newpage
\bibliographystyle{unsrt}
\bibliography{specspike}

\end{document}

